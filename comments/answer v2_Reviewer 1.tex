\documentclass{article}
\usepackage[utf8]{inputenc}
\usepackage{csquotes}
\usepackage[textsize=footnotesize]{todonotes}
\usepackage{subcaption}
% \usepackage{url}
\PassOptionsToPackage{hyphens}{url}\usepackage{hyperref}


\urlstyle{same}

\setlength{\parindent}{0pt}

\begin{document}

Dear Reviewer,

\vspace{0.25in}

Thank you for considering our manuscript for publication and for providing constructive feedback.
Hereunder you will find our detailed replies to all Your comments.
The changes are highlighted in the output of the latexdiff file attached to this cover letter.


\paragraph{Issue 1:}
\begin{displayquote}
"Since, one of the study aim was to reduce the data, therefore, by analysing ECG signals visually and by experiments we set the threshold of 2 which means all the data points associated with frequency more than two i.e., high QRS peaks (downward in lead V1) will be discard because the signals can classify through the base patterns." What can be understood that all frequency components above 2 are reduced to 0. However, the authors do not specify whether they mean 2 Hz (the unit is important). In this part of the paper there is no reference to the frequency band that the ECG signal covers. In the situation, as presented by the authors, then above a frequency of 2 Hz there are no longer any components derived from the ECG signal, as can be seen in the spectrograms. There are also reservations about the spectrograms presented, which have neither a time axis nor a frequency axis let alone the values that would be presented.
\end{displayquote}


\paragraph{Issue 1a:}
\begin{displayquote}
What can be understood that all frequency components above 2 are reduced to 0. However, the authors do not specify whether they mean 2 Hz (the unit is important). In this part of the paper there is no reference to the frequency band that the ECG signal covers.
\end{displayquote}

\paragraph{Answer:}
Thank you for this issue. In frequency filtration, our aim was to reduce the sample size as much as possible. Since, the Anteroseptal Myocardial Infarction can be distinguised through ST elevation in lead V1, therefore, we considered a high cutoff frequency components only i.e., above than 2 Hz as shown in fig. 3(a). We did not find any official information about frequency bands at PTB-XL dataset repo page. \\
We mentioned the unit with digit 2 (please see algorithm 2, section '3.1.2. Frequency Filtration' (line No. 277 and 280) and apologize for that.  

\paragraph{Issue 1b:}
\begin{displayquote}
In the situation, as presented by the authors, then above a frequency of 2 Hz there are no longer any components derived from the ECG signal, as can be seen in the spectrograms. There are also reservations about the spectrograms presented, which have neither a time axis nor a frequency axis let alone the values that would be presented
\end{displayquote}

\paragraph{Answer:}
Thanks for asking. Axis are now added into the spectrograms. The said figure No. 3 is revised along with description as per instructions.
The explanatory text has been added as well at section `3.1.2. Frequency Filtration `, Line No. 290-301: \\\\  '\textit{''In the below Fig. 3, (a) is showing spectrogram before frequency filtration in which x-axis represents data segments which means each segment contains five data points with their corresponding frequencies at y-axis, while (b) is showing after operation with the same axis information. However, number of data segments reduced in (b) be-cause we dropped all the data points having frequency greater than 2 Hz instead of replacing them with zero values. We considered the data segments at x-axis instead of time because it gives more presentable and disguisable information (among both clas-ses) for the purpose of neural network training with improved accuracy as shown in results section. It is pertinent to mention here that, both x and y axis were completely omitted from all of figures during preparation of dataset with the aim of neural net-work model training because it can affect the overall accuracy.''}

\paragraph{Issue 2:}
\begin{displayquote}
Nor has it been explained what is the relationship of the threshold of 2 that high QRS peaks (downward in the V1 lead) will be rejected. One may ask what does frequency have to do with the amplitude of the QRS complex in the ECG signal? Improper filtering can reduce the amplitude of the QRS complex, leading to the loss of important diagnostic information.
\end{displayquote}

\paragraph{Answer:}
Thank you for this issue. Since, we were aimed to reduce the data through removal of high peaks from the V1 lead, therefore, the cutoff 2 Hz were choosen. When we visualized the spectrogram, we picked the top frequencies for this purpose which then applied as frequency filtration. Additionally, without QRS peaks, ST elevation was still available in the filterd signal which was enough to distinguish both normal and disease classes in terms of accuracy achieved. 

\paragraph{Issue 3:}
\begin{displayquote}
The explanations provided in the above-mentioned issues of reducing interference in the ECG signal cannot be accepted and affect the overall perception of the submitted work for review. I suggest reading the frequency characteristics of the ECG signal, for example, as presented in the book by Clifford Gari D, Azuaje Francisco, McSharry Patrick E, Eds: ``Advanced Methods `\&' Tools for ECG Data Analysis''.
\end{displayquote}

\paragraph{Answer:}
Thanks for this issue and reference suggestion. In the two articles referenced as [54] and [55] in the reference section of submitted paper, experiments taken by the authors and they support the wavelet transformation as a suitable method for reducing interference / signal analysis. Therefore, we beleive that WT is suitable for our work which is shown in accuracy obtained as a results.
A description has been added in section '3.1.1. Denoising' at line 246-249 with the text of: \\\\ 
 '\textit{''The through experiments by the authors of [54-55], gave us a motivation to apply the Wavelet transformation for our purposed work to remove the noise from the signals as shown in Algorithm 1, when it retains important diagnostic data [54]. ''}

\paragraph{Issue 4:}
\begin{displayquote}
Figure 2b,d and 3 present obtained spectrograms. In the cited paper [45] (Kang M, Shin S, Jung J, Kim YT. Classification of Mental Stress Using CNN-LSTM Algorithms with Electrocardiogram Signals. 564 Journal of Healthcare Engineering. 2021 Jun 7;2021:e9951905. Available from: https://www.hindawi.com/journals/jhe/2021/9951905/) also used spectrograms however these spectrograms are in color with the axes described. Why were the color spectrograms not obtained (presented) in the reviewed paper? 
\end{displayquote}.

\paragraph{Answer:}
Thanks for asking about spectrograms color. Color mapping can be changed, it only need to change cmap but according to our neural network architecture, we are passing a particular specified channel i.e., single, due to which we achieved higher accuracy with our proposed architecture. Additionally, the images are not in particular (or binarized), it represents several intensities or band maps based on gray color intensity. This color mapping is only for visualization purposes but in our case, it is specified as per performance of our neural network achitecture. Additionally, colored images holds more memory than grayscale and we are focused about memory usage in proposed work. The attempt was to obtain acceptable accuracy through spectrograms which was not possible on raw signals. 

\paragraph{Issue 5:}
\begin{displayquote}
Issue 8b from the previous review. I maintain that up-sampling the signal does not make sense, this part of the work I would skip.
\end{displayquote}

\paragraph{Answer:}
Thank you for the suggestion. 
We agreed that up-sampling gives lower accuracy than original signal. In our investigation we checked if interpolation is able to make our analysis better. We skipped the up-sampling from Fig. 8 and added the following text in section '4.4.4. Sampling Rate' at line 426-430:\\\\

 '\textit{'We checked, if interpolation (up-sampling) is able to make our analysis better. For 200 Hz we get 71.32\% accuracy, for 400 Hz, 800 Hz, 600 Hz and 1k Hz, we get 82.34\%, 66.32\%, 81\% and 64.21\% accuracy respectively. We concluded that, adding or deleting dummy data among the original signals cannot give performance equal to the original signals.'}

\vspace{0.25in}

Sincerely yours,\\
Muhammad Farhan Safdar\\
for the authors

\end{document}
