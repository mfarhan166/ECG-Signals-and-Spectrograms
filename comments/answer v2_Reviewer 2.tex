\documentclass{article}
\usepackage[utf8]{inputenc}
\usepackage{csquotes}
\usepackage[textsize=footnotesize]{todonotes}
\usepackage{subcaption}
% \usepackage{url}
\PassOptionsToPackage{hyphens}{url}\usepackage{hyperref}


\urlstyle{same}

\setlength{\parindent}{0pt}

\begin{document}

Dear Reviewer,

\vspace{0.25in}

Thank you for considering our manuscript for publication and for providing constructive feedback.
Hereunder you will find our detailed replies to all Your comments.
The changes are highlighted in the output of the latexdiff file attached to this cover letter.

\paragraph{Issue 1:}
\begin{displayquote}
minor element: the article still needs improvement in the way that data is presented (not sure of the accuracy of some data or the way the figures were obtained), the quality of figures and explanations of data presented.
\end{displayquote}

\paragraph{Answer:}
TODO:
Fig 1: caption - add description, like: , x-axis (time), y- amplitude, typical size (1 sec),
Fig.2 (a) sambles (number of data points, y-axis - check unit, (b) frequency in Hz, give description of x-axis units, what is time segment here, (c)
Fig.3,  units (Hz), unit of x-axis, why time segments before to 200, after to 100
Fig.6, remove string 'precision metrices', add accuracy (%), precision (%), loss (%)
caption: 'Data analytics and validation accuracy on both datasets: raw signals and spectrograms'

Fig. 7, caption: 'Learning rate evaluation on both datasets: raw signals and spectrograms'
Fig. 8, caption: 'ECG sampling rate influence (downsampling, upsampling) on algorithm quality'

please highlight the oryginal sampling rate on the picture



Thanks for the issue.
We modified Fig.1 description, now it is ;
We also modified Fig.2 description, ...

We modified Fig.3 description, as well, ...

The figures are obtained in 600 dpi (while according to journal guidelines, these should be minimum 300 dpi). However, there were some datapoints overlapping which are revised with explanations. 


\paragraph{Issue 2:}
\begin{displayquote}
- major problem: the novelty statement is still not clear (and even written in a not grammatical way in the abstract). This is also not present in the conclusions. Without this, the paper is more of an engineering application presentation than a scientific article. From what is presented the reviewer has doubts if something really new was achieved.
\end{displayquote}

\paragraph{Answer:}
Thanks for the issue. Novelty statement is revised as `The novelty in proposed work is representation of possible reduced data to convolutional neural network model in well distinguishable form i.e., Spectrograms through Short Time Fourier Transformation (STFT) instead of raw signals. The data reduction performed by frequency filtration by taking a certain cutoff value with which a simple architecture of CNN model showed high accuracy rather than to utilize the complex model which ultimately requires more momeory and computational power. `\\\\
Novelty statement is also included in the conclusion. 



\vspace{0.5cm}

Sincerely yours,\\
Muhammad Farhan Safdar\\
for the authors

\end{document}
