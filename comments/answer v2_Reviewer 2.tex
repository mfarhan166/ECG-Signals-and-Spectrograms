\documentclass{article}
\usepackage[utf8]{inputenc}
\usepackage{csquotes}
\usepackage[textsize=footnotesize]{todonotes}
\usepackage{subcaption}
% \usepackage{url}
\PassOptionsToPackage{hyphens}{url}\usepackage{hyperref}


\urlstyle{same}

\setlength{\parindent}{0pt}

\begin{document}

Dear Reviewer,

\vspace{0.25in}

Thank you for considering our manuscript for publication and for providing constructive feedback.
Hereunder you will find our detailed replies to all Your comments.
The changes are highlighted in the output of the latexdiff file attached to this cover letter.

\paragraph{Issue 1:}
\begin{displayquote}
minor element: the article still needs improvement in the way that data is presented (not sure of the accuracy of some data or the way the figures were obtained), the quality of figures and explanations of data presented.
\end{displayquote}

\paragraph{Answer:}
Thanks for the issue. We modified all images description which is now showing as under: \\\\

'\textit{
Fig 1 description (line 60): ''ECG Cardiac Cycle (image credit: Public Domain): x-axis (time), y-axis (amplitude in mV),     typical duration (1 sec)''\\\\
Fig 2 description  (line 110): ''(a) Normal ECG signals: x-axis (samples/datapoints), y-axis (frequency in Hz) (b) Raw signal transformed into spectrogram: x-axis (data segments), y-axis (frequency in Hz), (c) ASMI ECG signals: x-axis (samples/datapoints) (d) Raw signal transformed into spectrograms: x-axis (data segments), y-axis (frequency in Hz)''\\\\
Fig 3 description  (line 303): ''(a) Spectrogram before frequency filtration: x-axis (Data segments), y-axis (frequency in Hz), (b) Spectrogram after frequency filtration: x-axis (Data segments), y-axis (fre-quency in Hz)''\\\\
Fig 4 description  (line 338): ''Proposed model architecture with 1 input layer, 4 convolutional layers and 1 output layer''\\\\
Fig 5 description  (line 369): ''Memory Proportion on both datasets: raw signals and spectrograms''. '\textit{(paragraph text is also update, please see lines 371-376.)} \\\\
Fig 6 description (line 386): ''(a) Data Analytics: accuracy, precision and loss in log scale for both datasets: raw signals and spectrograms, (b) Validation accuracy: x-axis (no. of epochs), y-axis (accuracy) for both datasets: raw signals and spectrograms''. '\textit{(paragraph text is also update, please see lines 391-397.)}\\\\
Fig 7 description (line 407): ''Learning rate evaluation on both datasets: raw signals and spectrograms: x-axis (learning rate), y-axis (accuracy) ''. '\textit{(paragraph text is also update, please see lines 410-413.}\\\\
Fig 8 description (line 422): ''ECG sampling rate influence (down sampling) on algorithm quality: x-axis (sampling rate), y-axis (accuracy)''. '\textit{(paragraph text is also update, please see lines 425-430.)}\\\\\\
}

The figures are obtained in 600 dpi (as according to journal guidelines, it should be minimum 300 dpi). However, there were some datapoints overlapping which are revised in fig 7 (line 406). 


\paragraph{Issue 2:}
\begin{displayquote}
- major problem: the novelty statement is still not clear (and even written in a not grammatical way in the abstract). This is also not present in the conclusions. Without this, the paper is more of an engineering application presentation than a scientific article. From what is presented the reviewer has doubts if something really new was achieved.
\end{displayquote}

\paragraph{Answer:}
Thanks for the issue. Novelty statement is revised as under: \\\\


'\textit{'The novelty of proposed work is using spectrograms instead of raw signals. Spectrograms could be easily reduced not to consider frequencies with no EKG information. Moreover, spectrogram calculation is time efficient through Short Time Fourier Transformation (STFT). Therefore convolutional neural network models have reduced data in the well-distinguishable form. The data reduction was performed by frequency filtration by taking a specific cutoff value. This step makes a simple architecture of the convolutional neural network (CNN) model shows high accuracy. Our approach reduces memory usage and computational power through not using complex CNN models.'}

Novelty statement is included in abstract and conclusion.

\vspace{0.5cm}

Sincerely yours,\\
Muhammad Farhan Safdar\\
for the authors

\end{document}
