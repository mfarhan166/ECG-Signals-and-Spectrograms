\documentclass{article}
\usepackage[utf8]{inputenc}
\usepackage{csquotes}
\usepackage[textsize=footnotesize]{todonotes}
\usepackage{subcaption}
% \usepackage{url}
\PassOptionsToPackage{hyphens}{url}\usepackage{hyperref}


\urlstyle{same}

\setlength{\parindent}{0pt}

\begin{document}

Dear Reviewer,

\vspace{0.25in}

Thank you for considering our manuscript for publication and for providing constructive feedback.
Hereunder you will find our detailed replies to all Your comments.
The changes are highlighted in the output of the latexdiff file attached to this cover letter.

\paragraph{Issue 1:}
\begin{displayquote}
The abstract gives a good sense of the article and allows even a person not familiar with the field to follow. However, the structure of the abstract where each chapter is described is problematic. Usually, the abstract is a short description of the article so the reader can get a sense of it without presenting results but with a strongly stated novelty of the paper. It is suggested to rewrite the abstract.
\end{displayquote}

\paragraph{Answer:}
Thanks for the issue. The abstract has been revised with novelty, with the text of '\textit{The non-invasive electrocardiogram (ECG) signals are useful in heart condition as-sessment and are found helpful in diagnosing cardiac diseases. However, traditional ways i.e., medical consultation and machine learning models require effort, knowledge, and time to interpret the ECG signals due to large amount of data and complexity. Neural networks have shown to be efficient recently in interpreting the biomedical signals including ECG and EEG. The novelty in proposed work is representation of data to convolutional neural network model in well understood i.e., Spectrograms. Moreover, frequency filtration was applied through Short Time Fourier Transoformation (STFT) with purpose of reducing data at possible limit at which a simple architecture of CNN model showed high accuracy rather than to utilize the complex model which ultimately requires more momeory and computational power. In this study, a diverse approach adopted by acquiring spectrograms using STFT as an input to convolutional neural network model. A large publicly available PTB-XL dataset was utilized, and two datasets were prepared i.e., spectrograms and raw signals for binary classification. Signal denoising, unnecessary frequency filtration and Short Time Fourier Transformation were applied to generate and process the spectrograms. Further, up and down sampling of the signals were performed at various points and accuracies attained. The highest accuracy of 99.06\% achieved by our proposed approach which reflects spectrograms are better than the raw signals. The software, developed in Python, is available freely on http:// https://github.com/mfarhan166/ECG-Signals-and-Spectrograms under MIT license.}' at lines 16-35

\paragraph{Issue 2:}
\begin{displayquote}
The authors are referencing mostly new publications, which is the advantage of state-of-the-art analysis. However, some sentences and specific statements are not connected to specific publications.
Additionally, most of the introduction is focused on ECG signal denoising and the part about ECG itself and its typical application is short. This is Therefore, the introduction would use some examples of medical usage of ECG for specific measurements like, hypoxic pregnancy conditions in-utero monitoring (e.g doi: 10.3934/mbe.2021250), fetal heart rate tracings (doi: 10.1007/s00404-019-05151-7), detection of heart disorders (M. Suganthy, Analysis of R-peaks in fetal electrocardiogram to detect heart disorder using fuzzy clustering, in IEEE 5th International Conference for Convergence in Technology (I2CT), (2019), 1-4.) and many other. The introduction would use some real-life examples. Suggest improving. This is the only drawback of quite a good introduction.
\end{displayquote}

\paragraph{Answer:}
Thanks for raising this issue. The real life examples of ECG are added as mentioned below.
ECG is useful in finding various heart related diseases including cardiovascular disorder [48], heart bundle branch block [49], ST-T ischemic changes for coronary heart disease [50], atrial fibrilliations due to disordered heart rhythm [51], left ventriuclar hypertrophy [52] and acute pericarditis [53]. \\\\

The useful applications of ECG are added in section `introduction' with the text of '\textit{ECG is useful in finding various heart related diseases including cardiovascular disor-der [48], heart bundle branch block [49], ST-T ischemic changes for coronary heart disease [50], atrial fibrillations due to disordered heart rhythm [51], left ventricular hypertrophy [52] and acute pericarditis [53]. }' at lines 53-56

\paragraph{Comment 3:}
\begin{displayquote}
The related works chapter is very good and presents a comprehensive and detailed analysis of what has been done in the field.
\end{displayquote}

\paragraph{Answer:}
Thanks for the appreciation.

\paragraph{Issue 4:}
\begin{displayquote}
The methodology chapter starts quite abruptly without any introduction to the matters in this chapter.
\end{displayquote}

\paragraph{Answer:}
Thanks for the pointing out. The approaches in methodology is section is introduced in one short paragraph as given below:

The ECG signals may contain various type of noises i.e., baseline wander, power line interference which produces some additional frequency components i.e., low, and high frequencies and can affect the classification accuracy. Therefore, wavelet trans-formation was considered for denoising the signals. Similarly, Fourier Transformation divides the biomedical signals into time-frequency domain, through which an un-wanted frequency can be filtered with the purpose of data size reduction. The key ad-vantage of reducing data size is to present the complete data into less resolution image i.e., spectrogram which helps in memory and computational management. Spectro-grams depicts the data into more distinguishable form than the raw signals i.e., recent studies [38-40] reveals that the data feed as spectrograms into neural network resulted in better accuracy which urged us to produce spectrograms using STFT.  \\\\

A short introduction is added in section `methodology' with the text of '\textit{The ECG signals may contain various type of noises i.e., baseline wander, power line interference which produces some additional frequency components i.e., low, and high frequencies and can affect the classification accuracy. Therefore, wavelet trans-formation was considered for denoising the signals. Similarly, Fourier Transformation divides the biomedical signals into time-frequency domain, through which an un-wanted frequency can be filtered with the purpose of data size reduction. The key ad-vantage of reducing data size is to present the complete data into less resolution image i.e., spectrogram which helps in memory and computational management. Spectro-grams depicts the data into more distinguishable form than the raw signals i.e., recent studies [38-40] reveals that the data feed as spectrograms into neural network resulted in better accuracy which urged us to produce spectrograms using STFT. }' at lines 222-232

\paragraph{Issue 5:}
\begin{displayquote}
It must be pointed out that some personal pronouns are being used which is not the correct form for a journal paper. Occurrences of “we” (e.g. line 204, 305, 355…and many others), our (e.g.lines $204,217,233 \ldots$ and many others) must be changed to impersonal forms.  In the current form, the language sometimes looks like a student thesis, not a scientific paper. It is especially visible in Tab.2 and 3 with the field name “Our proposed Work”
\end{displayquote}

\paragraph{Answer:}
Thanks for raising the point. The word like `we' and `our' are removed and sentence are slightly changed at various lines and mentioned tables.

\paragraph{Issue 6:}
\begin{displayquote}
Fig. 2 text is not visible, please improve it.  Fig/3 the same comment. Fig.4 please enlarge the values hardly visible. Fig.5 would be also better if enlarged. Additionally, from an editing point of view, it is suggested to unify fonts, font sizes, colours etc. Look at fig.6 which looks somehow different from previous figures.
\end{displayquote}

\paragraph{Answer:}
Thanks for this issue. The images font size enhanced.

\paragraph{Issue 7:}
\begin{displayquote}
The conclusions look good. However, again multiple uses of personal pronouns make it really unprofessional. Additionally, suggest presenting stronger the novelty of the paper.

\end{displayquote}

\paragraph{Answer:}
Thanks for raising the point. Use of personal pronouns removed from the whole paper and sentences are adjusted accordingly. \\\\

The novelty is described in the section `abstract' with the text of '\textit{The novelty in proposed work is representation of data to convolutional neural network model in well understood i.e., Spectrograms. Moreover, frequency filtration was applied through Short Time Fourier Transoformation (STFT) with purpose of reducing data at possible limit at which a simple architecture of CNN model showed high accuracy rather than to utilize the complex model which ultimately requires more momeory and computational power.}' at lines 21-26

\vspace{0.5cm}

Sincerely yours,\\
Muhammad Farhan Safdar\\
for the authors

\end{document}
